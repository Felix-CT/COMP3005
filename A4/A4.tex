\documentclass[a4 paper]{article}
% Set target color model to RGB
\usepackage[inner=2.0cm,outer=2.0cm,top=2.5cm,bottom=2.5cm]{geometry}
\usepackage{setspace}
\usepackage[rgb]{xcolor}
\usepackage{verbatim}
\usepackage{subcaption}
\usepackage{amsgen,amsmath,amstext,amsbsy,amsopn,tikz,amssymb,tkz-linknodes}
\usepackage{fancyhdr}
\usepackage[colorlinks=true, urlcolor=blue,  linkcolor=blue, citecolor=blue]{hyperref}
\usepackage[colorinlistoftodos]{todonotes}
\usepackage{rotating}
%\usetikzlibrary{through,backgrounds}
\hypersetup{%
pdfauthor={Ashudeep Singh},%
pdftitle={Assignment 4},%
pdfkeywords={Tikz,latex,bootstrap,uncertaintes},%
pdfcreator={PDFLaTeX},%
pdfproducer={PDFLaTeX},%
}
%\usetikzlibrary{shadows}
% \usepackage[francais]{babel}
\usepackage{booktabs}
\newcommand{\ra}[1]{\renewcommand{\arraystretch}{#1}}

\newtheorem{thm}{Theorem}[section]
\newtheorem{prop}[thm]{Proposition}
\newtheorem{lem}[thm]{Lemma}
\newtheorem{cor}[thm]{Corollary}
\newtheorem{defn}[thm]{Definition}
\newtheorem{rem}[thm]{Remark}
\numberwithin{equation}{section}

\newcommand{\homework}[6]{
   \pagestyle{myheadings}
   \thispagestyle{plain}
   \newpage
   \setcounter{page}{1}
   \noindent
   \begin{center}
   \framebox{
      \vbox{\vspace{2mm}
    \hbox to 6.28in { {\bf COMP 3005:~Database Management Systems \hfill {\small (#2)}} }
       \vspace{6mm}
       \hbox to 6.28in { {\Large \hfill #1  \hfill} }
       \vspace{6mm}
       \hbox to 6.28in { {\it Instructor: {\rm #3} \hfill Name: {\rm #5}, ID: {\rm #6}} }
       %\hbox to 6.28in { {\it TA: #4  \hfill #6}}
      \vspace{2mm}}
   }
   \end{center}
   \markboth{#5 -- #1}{#5 -- #1}
   \vspace*{4mm}
}

\newcommand{\problem}[2]{~\\\fbox{\textbf{Q #1}}\hfill (#2 points)\newline\newline}
\newcommand{\subproblem}[1]{~\newline\textbf{(#1)}}
\newcommand{\D}{\mathcal{D}}
\newcommand{\Hy}{\mathcal{H}}
\newcommand{\VS}{\textrm{VS}}
\newcommand{\solution}{~\newline\textbf{\textit{(Solution)}} }

\newcommand{\bbF}{\mathbb{F}}
\newcommand{\bbX}{\mathbb{X}}
\newcommand{\bI}{\mathbf{I}}
\newcommand{\bX}{\mathbf{X}}
\newcommand{\bY}{\mathbf{Y}}
\newcommand{\bepsilon}{\boldsymbol{\epsilon}}
\newcommand{\balpha}{\boldsymbol{\alpha}}
\newcommand{\bbeta}{\boldsymbol{\beta}}
\newcommand{\0}{\mathbf{0}}



\begin{document}
\homework{Assignment \#4}{Due: Friday November 19, 2021 (11:59 PM)}{Ahmed El-Roby}{}{Félix Cardinal Tremblay}{101141593}
\textbf{Instructions}: Read all the instructions below carefully before you start working on the assignment, and before you make a submission.
\begin{itemize}
    \item The accepted format for your submission is pdf only.
    \item If you use the tex file, make sure you edit line 28 to add your name and ID. Only write your solution and do not change anything else in the tex file. If you do, you will be penalized.
    \item \item Late submissions are allowed for 24 hours after the deadline above with a penalty of 10\% of the total grade of the assignment. Submissions after more than 24 are not allowed.
\end{itemize}

\problem{1:}{3}
In class, we showed that functional dependencies are transitive. That is, if $X \rightarrow Y$ and $Y \rightarrow Z$, then $X \rightarrow Z$. Assume a new proposed rule: If $X \rightarrow Y$ and $Z \rightarrow Y$, then $X \rightarrow Z$. Prove that this rule is incorrect.\\\\
Here is a counter example:
\begin{center}
  \begin{tabular}{|c|c|c|}
   \hline
    X & Y & Z\\ [0.5ex]
   \hline\hline
   a & 1 & b\\
   \hline
   a & 1 & c\\
   \hline
   b & 1 & a\\
   \hline
  \end{tabular}
\end{center}
$X\rightarrow Y$ is valid as there is no situation where the $X$ values are the same, but the values of $Y$ is different.\\
$Z\rightarrow Y$ is valid as there is no situation where the $Z$ values are the same, but the values of $Y$ is different.\\
$X\rightarrow Z$ is invalid, as we have the tuples $(a,b),(a,c)$, so $X$ cannot be functionally dependent on $Z$



\problem{2:}{3}
How can you use functional dependencies to represent the constraint that a relationship between two entity sets $X$ and $Y$ is one-to-many from $X$ to $Y$.\\\\
You would denote it as $Y\rightarrow X$. This is because multiple values of $Y$ can have the same value of $X$, without the constraint being violated. On the other hand, if we were to have different values of $X$ for the same value of $Y$, the constraint (and the relationship) would be violated.


\problem{3:}{8}
Consider the following relation $R = \{A, B, C, D, E\}$ and the following set of functional dependencies \\$F = \{
\\A \rightarrow BC\\
CD \rightarrow E\\
B \rightarrow D\\
E \rightarrow A\}$\\
Compute $B^{+}$. Is $R$ in BCNF? If not, give a lossless decomposition of $R$ into BCNF. Show your work for all previous questions.\\\\
\begin{align*}
  &\text{result} = B\\
  &B\rightarrow D: \text{ result} = BD
\end{align*}
$R$ is not in BCNF. This is because for $R$ to be in BCNF, we need the all of the values on the left side of the arrows to be superkeys. Since $B$ is not a superkey, it means that $R$ cannot be in BCNF.\\\\
We will begin by listing out all of the closures of the Left hand side of the arrows.
$A^+$:
\begin{align*}
  &\text{result} = A\\
  &A\rightarrow BC: \text{ result} = ABC\\
  &B\rightarrow D: \text{ result} = ABCD\\
  &CD\rightarrow E: \text{ result} = ABCDE
\end{align*}
and thus $A^+\subseteq R$.\\
$(CD)^+$:
\begin{align*}
  &\text{result} = CD\\
  &CD\rightarrow E: \text{ result} = CDE\\
  &E\rightarrow A: \text{ result} = ACDE\\
  &A\rightarrow BC: \text{ result} = ABCDE
\end{align*}
and thus $(CD)^+\subseteq R$.\\
$E^+$:
\begin{align*}
  &\text{result} = E\\
  &E\rightarrow A: \text{ result} = AE\\
  &A\rightarrow BC: \text{ result} = ABCE\\
  &B\rightarrow D: \text{ result} = ABCDE\\
\end{align*}
and thus $(E)^+\subseteq R$.\\
So, finally we have:
\begin{align*}
  &A^+: ABCDE\\
  &CD^+: ABCDE\\
  &E^+: ABCDE\\
  &B^+: BD
\end{align*}

So, we need to create two relation such that A, CD, E, and B are primary keys, and that the intersection of the two relations is a primary key of at least one of the relations. We know that for $B$ to be a primary key, we need it to be in a relation that only contains $B$ and $D$. Therefore, we know that $R_1=\{B,D\}$. Next, since the closure of the other three attributes is $R$, it does not matter which values we choose, as our remaining values determine all values. So, we can choose it as the following: $R_2=\{A,B,C,E\}$. Since $R_1\cup R_2={B}$, and $B$ is a superkey for $R_2$, we know it must be lossless.\\\\
Since $A^+, CD^+$ and $E^+$ are superkeys for $R_2$ and $B$ is a superkey for $R_1$, we know the decomposition must be in BCNF.






\problem{4:}{4}
Give a lossless, dependency-preserving decomposition into 3NF of schema $R$ in Q3.\\\\

We start with the following functional dependencies:
\\$F = \{
\\A \rightarrow BC\\
CD \rightarrow E\\
B \rightarrow D\\
E \rightarrow A\}$\\
We want to find the canonical cover, that is, $F_c$.\\\\
First, we will see if there are any extraneous ttributes in the left side. we have $CD\rightarrow E$, and so we want see if we can imply $C\rightarrow E$ or $D\rightarrow E$ from $C^+$ or $D^+$.\\\\
Since ther are no other depencies with $D$ or $C$ on the left side, both $D^+$ and $C^+$ only contan $D$ an $C$ respectively. Next, we'll check if there  are any extraneous attributes on the right side. There is only one dependency with two attributes on the left side, which is $A\rightarrow BC$. Therefore we need to calculate $B^+$, since $C^+$ is already calculated.\\
We can see that $B\rightarrow D$, and so $B^+=BD$. However, there are no other relations for which $B$ is on the left hand side, and so we cannot infer anything else. Thus, since we have no way of showing $A\rightarrow B$ or $A\rightarrow C$, there cannot be any extraneous attributes on the right hand side, and so $F=F_c$. Now, using this canonical cover, we want to calculate 3NF.













The dependencies $A\rightarrow BC$ and $E\rightarrow A$ can be replaced by $E\rightarrow ABC$. Thus, our current $F_c$ is:\\
$\{
\\E \rightarrow ABC\\
CD \rightarrow E\\
B \rightarrow D\}$\\

Next, we notice that $E \rightarrow ABC$ and $CD \rightarrow E$ can be replaced by $CD\rightarrow ABCE$, so $F_c$ is:\\
$\{
\\CD \rightarrow ABCE\\
B \rightarrow D\}$\\

%THIS COULD DEFINITELY BE WRONG
%Since we have $C$ on both sides, we can then show $CD \rightarrow ABCED=\rightarrow ABED$

Since there are no extraneous attributes in the functional dependencies:
$F_c=\{B\rightarrow D, CD\rightarrow ABCE$\\
After the first loop of the algorithm, we would get the following relations:\\
$R_1:\{A,B,C,D,E\}$, $R_2:\{B,D\}$
Then, because $R_2\subseteq R_1$, we can delete $R_2$.

%I computed the canonical cover, but I dind't put it in 3nf

\problem{5:}{4}
Assume the following decomposition of $R$ in Q3: $R_{1}(A, B, E)$ and $R_{2}(C, D, E)$. Is this decomposition lossy or lossless? Why? Show your work in detail.\\\\
For the decomposition of a relation to be lossless, we need one of the two following conditions to be true: $R_1 \cap R_2\rightarrow R_1$, or $R_1 \cap R_2\rightarrow R_2$.
Since $R_1\cap R_2=\{\}$, this means that the decomposition must be lossy.



\problem{6:}{22}
Consider the following relation $R(A, B, C, D, E, G)$ and the set of functional dependencies \\$F = \{
\\A \rightarrow BCD\\
BC \rightarrow DE\\
B \rightarrow D\\
D \rightarrow A\}$\\

\noindent Note: Show the steps for each answer.

\subproblem{a} Compute $B^{+}$. \indent (4 points)\\
\begin{align*}
  &\text{result} = B\\
  &B\rightarrow D: \text{ result} = BD\\
  &D\rightarrow A: \text{ result} = ABD\\
  &A\rightarrow BCD: \text{ result} = ABCD\\
  &BC\rightarrow DE: \text{ result} = ABCDE\\
\end{align*}
so, our final result is $B^+=\{A,B,C,D,E\}$




\subproblem{b} Prove (using Armstrong's axioms) that $AG$ is superkey. \indent (4 points)\\
To find if $AG$ is a superkey, we must first find $AG^+$ and find if ir contains all attributes in $R$. We can start with the following:

$$A\rightarrow BCD$$
By the decomposition rule, we know that if $A\rightarrow BCD$, then $A\rightarrow B, A\rightarrow C$, and $A\rightarrow D$ are all valid. This means that by the union rule, we can show $A\rightarrow BC$, and then by the transitivity rule, as can show $A\rightarrow DE$. Then, finally, by the union rule, we can combine the two previous ones to finish with $A\rightarrow BCDE$.\\\\
Since we have the functional dependecy $A\rightarrow BCDE$, and we know $AG$, this means that $AG+=\{A, B, C, D, E, G\}$. Therefore, since $AG+$ contains all elements in $R$, $AG$ is a superkey for $R$.

\subproblem{c} Compute $F_{c}$. \indent (6 points)\\
%Since $B\rightarrow D$ and $D\rightarrow A$, we can use the transitivity rule to show $B\rightarrow A$, and then use the union rule to show $B\rightarrow AD$. This means our current $F_c$ will look like the following:
%\\$F_c = \{
%\\A \rightarrow BCD\\
%BC \rightarrow DE\\
%B \rightarrow AD}$\\

We begin with the following:\\
$F=\{\\
A \rightarrow BCD\\
BC \rightarrow DE\\
B \rightarrow D\\
D \rightarrow A\}$\\

We will check to see if $D$ is extraneous in $BC\rightarrow DE$.\\
$F'=\{\\
A\rightarrow BCD\\
BC\rightarrow E\\
B\rightarrow D\\
D\rightarrow A\}$
Now, we need to check if $BC\right D$ can be implied by $F'$. Since $B\rightarrow D$, we know $BC\rightarrow D$.
Next, we can see that $B^+=\{A,B,C,D,E\}$, as we calculated in the previous question, so since $D\in B^+$, $D$ is extraneous. Thus, currently:\\
$F_c=\{\\
A\rightarrow BCD\\
BC\rightarrow E\\
B\rightarrow D\\
D\rightarrow A\}$
\\\\
Next, we need to check if there are any other possible extraneous attributes. We can see that since there are no dependencies with $C$ as an attribute in the left hand side, $C$ wll not be extraneous.\\
Additionally, we can notice that $D$ may be extraneous in $A\rightarrow BCD$. To check, we will define:\\
$F'=\{\\
A\rightarrow BC\\
BC\rightarrow E\\
B\rightarrow D\\
D\rightarrow A\}$\\
Then, we will compute $A^+$
From the first dependency, we know $A^+$ will have $ABC$. Then, from the 2nd dependency, since $A\rightarrow BC$ and $BC\rightarrow E$, by the transitivity rule we can infer that $A\rightarrow E$, so we will have $ABCE$. Next, by the third dependency, since we know $A\rightarrow B$, and $B\rightarrow D$, by transitivity, $D$ will be in $A^+$. Since $A^+$ contains $D$, it must be extraneous. Therefore, our current canonical cover will be:\\
$F_c=\{\\
A\rightarrow BC\\
BC\rightarrow E\\
B\rightarrow D\\
D\rightarrow A\}$\\

Now, since we can quite trivially see that $C$ is not in $B^+$ and $B$ is not in $C^+$, we know that there must not be any more extraneous attributes. Therefore, our final canonical cover will be:\\
$F_c=\{\\
A\rightarrow BC\\
BC\rightarrow E\\
B\rightarrow D\\
D\rightarrow A\}$\\






% Now, since we know $A\rightarrow BCD$, we can use the decoposition rule to show $A\rightarrow BC$. Then, with the transitivity rule we can show $A\rightarrow DE$, because $BC\rightarrow DE$. Therefore, by using the union rule, our current $F_c$ will be:
% \\$F_c = \{
% \\A \rightarrow ABCDE\\
% B \rightarrow AD$\}\\
% Then, with the decomposition rule we can obtain $A\rightarrow B$. Since we can already have $A\rightarrow AD$ from the decomposition + augmentation rule on the first depdency, we can simply show that $F_c=\{A\rightarrow ABCDE\}$
%This seems wrong, since you're losing the extra information right?


\subproblem{d} Give a 3NF decomposition of the given schema based on a canonical cover. \indent (4 points)



\subproblem{e} Give a BCNF decomposition of the given schema based on $F$. Use the first functional dependency as the violator of the BCNF condition. \indent (4 points)\\



\problem{7:}{6}
Given the following set of functional dependencies:\\
$A \rightarrow BC$\\
$B \rightarrow AC$\\
$C \rightarrow AB$\\
Show that it is possible to find more than one unique canonical cover for this set.\\\\

We will first consider the functional dependency $B\rightarrow AC$.\\
We know that $C\in AC$, and $(F-\{B\rightarrow AC\})\cup\{B\rightarrow A\}$, because $B\rightarrow A$ and $A\rightarrow C$. Therefore, $C$ must be an extraneous attribute in the dependency $B\rightarrow AC$. Thus,\\
$F_c=\{\\
A\rightarrow BC\\
B\rightarrow A\\
C\rightarrow AB\}$\\
Next, we will consider the functional depdency $A\rightarrow BC$. We know that $B\in BC$, and and $(F-\{A\rightarrow BC\})\cup\{A\rightarrow C\}$, because $A\rightarrow C$ and $C\rightarrow B$. Therefore, $B$ must be an extraneous attribute in the dependency $A\rightarrow BC$. Thus,\\
$F_c=\{\\
A\rightarrow C\\
B\rightarrow A\\
C\rightarrow AB\}$\\
Next, we will consider the functional depdency $C\rightarrow AB$. We know that $A\in AB$, and and $(F-\{C\rightarrow AB\})\cup\{C\rightarrow B\}$, because $C\rightarrow B$ and $B\rightarrow A$. Therefore, $A$ must be an extraneous attribute in the dependency $C\rightarrow AB$. Thus,\\
$F_c=\{\\
A\rightarrow C\\
B\rightarrow A\\
C\rightarrow AB\}$\\\\
We have finished computing the first possible canonical cover, and we will now compute the second:\\\\
We will first consider the functional dependency $B\rightarrow AC$.\\
We know that $A\in AC$, and $(F-\{B\rightarrow AC\})\cup\{B\rightarrow C\}$, because $B\rightarrow C$ and $C\rightarrow A$. Therefore, $A$ must be an extraneous attribute in the dependency $B\rightarrow AC$. Thus,\\
$F_c=\{\\
A\rightarrow BC\\
B\rightarrow C\\
C\rightarrow AB\}$\\
We know that $C\in AC$, and $(F-\{A\rightarrow BC\})\cup\{A\rightarrow B\}$, because $A\rightarrow B$ and $B\rightarrow C$. Therefore, $C$ must be an extraneous attribute in the dependency $A\rightarrow BC$. Thus,\\
$F_c=\{\\
A\rightarrow B\\
B\rightarrow C\\
C\rightarrow AB\}$\\
Then, we know that $B\in AB$, and $(F-\{C\rightarrow AB\})\cup\{C\rightarrow A\}$, because $C\rightarrow A$ and $A\rightarrow B$. Therefore, $B$ must be an extraneous attribute in the dependency $C\rightarrow AB$. Thus,\\
$F_c=\{\\
A\rightarrow B\\
B\rightarrow C\\
C\rightarrow A\}$\\\\
Therefore, we have arrived at our second canonical cover, and shown that there can be more than one valid canonical covers for a set of dependencies.



% We will begin by examining $B\rightarrow AC$.\\
% We can see that by the decomposition rule, $B\rightarrow A$, and $B\rightarrow C$. By the transitivity and union rules, we then combine $B\rightarrow A$ with $A\rightarrow BC$ to have $B\rightarrow ABC$, which reduces to $B\rightarrow AC$. Therefore, we remove $A\rightarrow BC$ from $F_c$ Since all of its attributes are accounted for. Additionally, $C\rightarrow AB$ does not provide any new information, and so our final $F_c$ is $\{B\rightarrow AC\}$.\\\\
% Now, we will do the same process, but begin by examining $C\rightarrow AB$.\\
% By the decomposition rule, we can obtain $C\rightarrow A$ and $C\rightarrow B$. Next, we can use the transitivity and union rule with $A\rightarrow BC$, to obtain $C\rightarrow ABC$, which reduces into $C\rightarrow AB$, and remove $A\rightarrow BC from F_c$. Finally, $B\rightarrow AC$ does not provide any new information, and so it can be removed. Therefore, our second possile version of $F_c$ is $\{C\rightarrow AB\}$, and so there can be more than one unique canonical cover for this set.



\problem{8}{7}
Consider the schema $R = (A, B, C, D, E, G)$ and the set $F$ of functional dependencies:\\
$A \rightarrow BC$\\
$BD \rightarrow E$\\
$CD \rightarrow AB$\\
Use the BCNF decomposition algorithm to find a BCNF decomposition of $R$. Start with $A \rightarrow BC$. Explain your steps. Is this decomposition lossy or lossless? Is it dependency-preserving?\\\\
\begin{align*}
  &\text{result} = ABC\\
\end{align*}
We cannot infer $BD$ or $CD$ from our current result, so $R_1=\{A,B,C\}$.
Next, we can look at $BD\rightarrow E$. Again, knowing $BDE$ does not allow us to infer any more information, and so $R_2=\{B,D,E\}$.\\
Then, we have $CD\rightarrow AB$. Since we know $A$, we can add $A\rightarrow BC$ to our result. Therefore, we have



\problem{9:}{3}
As discussed in class, SQL does not support functional dependency constraints. But it supports materialized views. Assume that the DBMS maintains the materialized view immediately. Given a relation $R(W, X, Y, Z)$, how would you use materialized views to enforce the functional dependency $W \rightarrow Z$?\\\\
I would create a view that contains only $W$ and $Z$, where $W$ is the primary key.


\end{document}


\end{document}
